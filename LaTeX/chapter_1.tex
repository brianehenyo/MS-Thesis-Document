%%%%%%%%%%%%%%%%%%%%%%%%%%%%%%%%%%%%%%%%%%%%%%%%%%%%%%%%%%%%%%%%%%%%%%%%%%%%%%%%%%%%%%%%%%%%%%%%%%%%%%
%
%   Filename    : chapter_1.tex 
%
%   Description : This file will contain your Research Description.
%                 
%%%%%%%%%%%%%%%%%%%%%%%%%%%%%%%%%%%%%%%%%%%%%%%%%%%%%%%%%%%%%%%%%%%%%%%%%%%%%%%%%%%%%%%%%%%%%%%%%%%%%%


\Section{Research Description}
\label{sec:researchdesc}    %--note: labels help you with hyperlink editing (using your IDE)

This chapter discusses an overview of the current state of technology, the research objectives, scope and limitations, and its significance.

\subsection{Overview of the Current State of Technology}
\label{sec:overview}

Natural language processing systems use a set of knowledge base in order to do tasks such as text generation. But simple lexicons and large unstructured corpora may be insufficient as knowledge base of these systems. Storytelling, for instance, is a natural task for humans. Armed with a library of words, their meanings and their relationships, we combine words and events to tell stories about ourselves, our community, and our experiences. Thus computers must be provided with the same shared collection of common sense knowledge about the basic relationships between things and events that nearly every person knows in order for them to achieve a level of expressiveness same as humans and be able to understand the world that we talk about. Such knowledge are represented as conceptual relations defining the relationship between two or more concepts in real life.

Recent creative text generation systems such as \cite{Hong:2009} have utilized a semantic network representation of concepts on common sense knowledge to identify relationships of words in human puns in order to generate computer puns. Another system, Picture Books \cite{Solis:2009}, generates stories with morals for children ages 4 to 6, by using a semantic ontology, patterned after ConceptNet \cite{Liu:2004a}, containing conceptual knowledge about objects, activities, and their relationships in a child's daily life. The process of building and populating the Picture Books ontology required a lot of manual effort on the part of the proponents. Currently, the ontology contains 240 concepts and 369 relations, which were populated based on the themes that have been identified as relevant for the target age group.

Early Information Extraction (IE) systems have addressed the extraction of information from relatively small collections of well-structured documents such as newswire or scientific publications \cite{Muslea:1999}. More recently, IE systems are focused on extracting facts from structured and unstructured documents for a particular domain, such as legal documents \cite{Cheng:2008}.

Although IE systems are capable of recognizing entities within documents (e.g. `Renoir' is a `Person', `25 Feb 1841' is a `Date'), the relation between the entities (e.g., `Renoir' was born on `25 Feb 1841') is not extracted, thus generating incomplete information that may be needed by certain applications \cite{Banko:2008}. A variant of IE, Relation Extraction (RE), is the task of recognizing the assertion of a particular relationship between two or more entities in text.

The task of relation extraction is difficult, but relations such as hypernymy (IsA) and meronymy (PartOf) are often expressed using a small number of lexico-syntactic patterns \cite{Hearst:1992}. Using a sample set of 500 sentences selected at random from an IE training corpus, \citeA{Banko:2008} showed that many binary relationships are also consistently expressed using a compact set of relation-independent lexico-syntactic patterns.

The Artequakt project \cite{Alani:2003} also showed that it is possible to automatically acquire such relations from documents to populate an ontology. Working in the domain of artists, the Artequakt project identifies relations between entities of interest within sentences, following ontology relation declarations and lexical information. These relations are then used to populate an ontology with knowledge triples for use in the generation of biographies of artists.

To convey the ideas of a story, \citeA{Nakasone:2006} developed a storytelling ontology model by identifying relations between sentences in the story using the Rhetorical Structure Theory (RST) of \citeA{Mann:1987}. As \citeA{Knott:1994} pointed out, explicit and implicit relations hold between the sentences of a text, so that the content of one sentence might provide justification, elaboration or explanation for the content of another. These relations bind a text together to contribute to the overall comprehension of a story by the readers; for instance, whether understanding one text span (scene of a story) increases the reader's readiness to understand another scene, or whether understanding both spans allows the reader to recognize a particular semantic relation as holding between them. Certain discourse relations or cue phrases, such as ``but", ``so", ``although", ``more precisely" or ``for example", are used to signal explicit relations between text spans.

Although both IE and RE have achieved significant progress in extracting facts and concepts in the domains of newspapers \cite{Muslea:1999}, biographies \cite{Alani:2003}, and legal documents \cite{Cheng:2008}, limited work has been done on children's stories. Furthermore, since stories contain sequences of actions that characters perform or experience at various points in the story world, knowledge about how these events are ordered and the constraints under which they can occur must also be extracted.

\subsection{Research Objectives}
\label{sec:researchobjectives}

\subsubsection{General Objective}
\label{sec:generalobjective}

To develop a methodology that automatically identifies and extracts the relations between everyday concepts and objects from children's stories and store them in a semantic network to provide ontological knowledge for Picture Books.

\subsubsection{Specific Objectives}
\label{sec:specificobjectives}

\begin{enumerate}
   \item To collect a corpus of children's stories;
   \item To analyze the English sentence structures in the corpus;
   \item To derive a set of extraction patterns;
   \item To develop a representation for modelling relations of every object common in children's stories;
   \item To implement extraction rules using GATE for extracting conceptual relations automatically from the corpus, and;
   \item To validate the resulting conceptual relations extraction tool through comparison with a gold standard and integration with Picture Books
\end{enumerate}

\subsection{Scope and Limitations of the Research}
\label{sec:scopelimitations}

At least 30 children's stories were collected to form the input corpus for the extraction tool developed. These were then  modified to remove the dialogues and to clean of inconsistencies. English sentence structures in these stories were analyzed to identify the types of relations that are present. This information was then used to derive a set of patterns or templates for extracting conceptual relations. An existing extraction tool was used in utilizing the extraction patterns to automatically locate instances of a known relation in the corpus.

In terms of extraction patterns, one relation can be expressed in various ways in text. Consider the hypernymy (IsA) relation, wherein the following sentences are possible ways of expressing it:

	\noindent
	\hspace{1 in}\emph{The dog is a canine.} \\
	\hspace*{1 in}\emph{The dog is a kind of canine.} \\
	\hspace*{1 in}\emph{The dog, a canine, is �} \\
	\hspace*{1 in}\emph{The dog is a type of canine.}

But although lexico-syntactic extraction patterns can be mapped directly to relations, certain English sentence constructs require further analysis and decomposition in order to derive their corresponding relations. These include sentence structures containing conjunctions and embedded clauses, as shown in the examples below:

	\noindent
	\hspace{1 in}\emph{Cake is made of flour, sugar, \underline{and} butter.} \\
	\hspace*{1 in}\emph{The boy is singing \underline{and} the girl is dancing.} \\
	\hspace*{1 in}\emph{Anna, \underline{who is the queen}, went to the market, \underline{while} the king went to the mall.}

In line with the objectives of this research, all simple lexico-syntactic extraction patterns were handled given that they are from ConceptNet and the corpora at hand. The more complex sentences were manually modified into simpler sentences. Only then that the relations were extracted.

The extracted knowledge was stored in a representation model that can be used by NLP systems, in this case, the Picture Books story generator. It was patterned after the design of the Picture Books ontology, which is also patterned after the design of ConceptNet, to validate the presence of the appropriate relations against those identified from the collected corpus.

Since stories are sequences of events, their analysis necessitate the creation of new relations to represent sequences of events, temporal relations between events, as well as the constraints under which certain events may take place. For example, during testing, evaluators noticed that one of the generated stories of Picture Books occurred at an inappropriate time; specifically, the first segment of the story that introduces the day, the place, and the main character, contained the following text:

	\noindent
	\hspace{1 in}\emph{The evening was warm. Ellen the elephant was at the school. She went with Mommy Edna to the school.}

Since the knowledge base of Picture Books currently does not provide relations about when certain events can occur, the main character went to school in the evening. 

Sixteen (16) new semantic relations, resulting from analyzing the sample corpus, and from reviewing other works such as \cite{Mueller:2003} for modeling time and event occurrences, were created in this research. The main relations, mostly from ConcepNet and Picture Books, include IsA, PropertyOf, PartOf, MadeOf, EventForGoalEvent, EventForGoalState, EffectOf, EffectOfIsState, CapableOf, OftenNear, LocationOf and, UsedFor. Additional relations include Happens(e, t) which represents that a fluent \emph{f} holds at time \emph{t}, HasRole which represent character roles, RoleResponsibleFor which represent actions/events done by a specific role and lastly, Owns which represent ownership. 

\citeA{Alani:2003} noted that it is inevitable for duplicate and contradictory information to be extracted from the input corpus. But he further noted that handling such information is challenging for automatic extraction and ontology population approaches. In this research, only duplicate extractions were handled.

For evaluation purposes, the Picture Books' code base was not modified in any way to accommodate the additional relations and their corresponding entries in the ontology. The same goes for additional entries for existing Picture Books' relations. All modifications in Picture Books were done only in its database.

\citeA{Mueller:1999} also noted that ``story understanding goes beyond generating parse trees, disambiguating words, or filling templates, and includes the ability to answer arbitrary questions, generate paraphrases and summaries, fill arbitrary templates, make inferences, reason about the story, follow reasoning in the story, relate the story to general knowledge, and hypothesize alternative versions of the story." Thus, aside from having a huge collection of common sense knowledge, a computer system must also be able to ``make inferences about states and events not explicitly described in the text" \cite{Mueller:2003}, by performing common sense reasoning using knowledge about the world. This requires a multi-representational model of this knowledge for the various realms of space, time, needs and feelings to be built, and is beyond the scope of the current proposal.

Manual validation with the help of a linguist was not utilized. Instead, a gold standard was used to evaluate the extracted relations' accuracy and completeness. Aside from this, the extracted relations were automatically validated through the generated stories of Picture Books. 

\subsection{Significance of the Research}
\label{sec:significance}

Researches in the field of natural language processing (NLP) seek to find ways to make human-computer interaction more fluent. But human-computer communication is hampered by the lack of a shared collection of common sense knowledge that people rely on when they communicate in order to understand each other. In order to make computers achieve the same level of expressiveness as humans, we must give them ``a common language with richness that more closely approaches that of the human language" \cite{Niles:2001}.

Although dedicated IE systems have been developed to extract information from various domains, this research is a first step towards extracting relations from children's stories. Storytelling is a natural and familiar means of conveying information and experience to listeners \cite{Nakasone:2006}, thus justifying the selection of this domain for the proposed research.

The knowledge base derived from extracting entities, concepts, and events can then be used for various applications. One such application is in story generation. Picture Books can use the knowledge base to generate more variants as well as longer stories. Currently, Picture Books generates stories containing 24 to 30 sentences, which not only vary according to the age of the reader, but is also dependent on the available relations between two concepts. Story generating applications can in turn be used for both educational and entertainment purposes.

In education, \citeA{Riedl:2004} applied narrative generation techniques to generate historical fictions for teaching history, which they defined as ``the chronological record of significant events". \shortciteA{Lester:2007} explored integrating narratives into learning environments that teach microbiology to provide an ``adaptive, effective pedagogy that is both motivating and meaningful".

In entertainment, story generation is applied to develop interactive fiction systems. \citeA{Montfort:2009} defines interactive fiction as ``a venerable thread of creative computing and a literary art". His Curveship project uses NLP techniques to create narratives in the virtual world, where the user directs the possible flow of the story. For his knowledge base, Montfort utilized a tree representation that describes the possible sequences of events and the relationship of events to one another, as well as models of objects in the virtual world. A similar system in the game area, Fa�ade \cite{Mateas:2003} is a 3D interactive drama that makes use of artificial intelligence techniques to allow players to interact with the characters in the story by playing as one of the characters and typing textual commands that affect the flow and the outcome of the game (story). \citeA{Young:2008} is also exploring the development of computational models to generate narratives for 3D virtual game environments, which are being considered as alternative approach to promote learning.

Story understanding system can also benefit from using the knowledge base. Story understanding requires an enormous amount of common sense knowledge, thus the question and answering system of \citeA{Mueller:2007} has a limited scope focusing on modeling the spatial and temporal aspects of narratives involving one or two characters dining in a restaurant. He employed a combined technique using IE to extract key information about dining episodes, and common sense reasoning to build models of the dining episodes. The model is limited to only a single spatial layout consisting of the street, the dining room, and the kitchen, and further work can be done to extract information about the spatial layout from the text, and use this to construct models of room-scale space.

\subsection{Research Methodology}

This section discusses the systematic approach that was performed in order to accomplish the objectives of this research. Despite the chronological order of each phase, some were performed in parallel while some were backtracked to address some issues. Documentation and regular consultations with the adviser and work group members were done throughout the course of the research.

\subsubsection{Requirements Analysis}

After an extensive analysis of existing relevant implementations and literature, the researcher first identified the different conceptual/semantic relations that can be applied to the children's story domain. Then, redundancies in the existing relations of Picture Books was noted. Afterwards, an open-source tool was identified to aid in the extraction of relations. Lastly, en evaluation criteria was identified to validate the resulting ontology.

\subsubsection{Data Gathering}

During this stage, the children's story corpus consisting of 30 physical copies of children's stories were gathered and digitally encoded. Modifications were done, such as transforming dialogues into declarative sentences, to address a different scenario. Please see Appendix C for a sample of this modification. Lastly, the extraction templates were aligned with ConceptNet, while some were created based on sentence structures in the gathered children's stories.

\subsubsection{Architectural Design}

In this stage, the different components of the open-source tool were identified to enable a thorough and complete extraction. This includes a part-of-speech tagger, named-entity identifier, and gazetteer, among others. Other resources to be utilized were also identified. Furthermore, the data structures which will represent the semantic relations in Picture Books will also be analyzed and designed. 

\subsubsection{Implementation}

The final selection of the open-source tool components and their respective customizations were done at this stage. The primary focus was given to the finalization of extraction templates and their actual implementation into rules. Lastly, extracted relations were parsed and collated.

\subsubsection{Testing}

Testing was done to ensure the quality of the extracted relations. Unit testing for each component was performed. After doing so, integration testing was performed to verify that each component receives the correct input from the previous component and generates the appropriate result for use by subsequent components. 

Lastly, the outputs of the system were mainly evaluated by generating stories out of Picture Books. These output stories are then put through the evaluation criteria identified during the requirements analysis phase.

\subsubsection{Documentation}

Throughout the entire project, documentation was done to track progress. This was also to ensure that any changes and implementations in the requirements of the study will be reflected in the documents.

\subsection{Calendar of Activities}

Tables \ref{tab:timetableactivities1}, \ref{tab:timetableactivities2} and \ref{tab:timetableactivities3} shows a Gantt chart of the activities. Each bullet represents approximately one week worth of activity. The overlapping activities ensure that any omissions and modifications will be changed immediately. 

%
%  the following commands will be used for filling up the bullets in the Gantt chart
%
\newcommand{\weekone}{\textbullet}
\newcommand{\weektwo}{\textbullet \textbullet}
\newcommand{\weekthree}{\textbullet \textbullet \textbullet}
\newcommand{\weekfour}{\textbullet \textbullet \textbullet \textbullet}

%
%  alternative to bullet is a star 
%
\begin{comment}
   \newcommand{\weekone}{$\star$}
   \newcommand{\weektwo}{$\star \star$}
   \newcommand{\weekthree}{$\star \star \star$}
   \newcommand{\weekfour}{$\star \star \star \star$ }
\end{comment}



\begin{table}[ht]   %t means place on top, replace with b if you want to place at the bottom
\centering
\caption{Timetable of Activities for the months of January to July 2010} \vspace{0.25em}
\begin{tabular}{|p{2in}|c|c|c|c|c|c|c|c|} \hline
\centering Activities (2010) & Jan & Feb & Mar & Apr & May & Jun & Jul \\ \hline
Requirements Analysis	   	& ~~~\weektwo & \weekfour & \weekfour & \weekfour &  &  &  \\ \hline
Data Gathering			   	&  &  & ~~~\weektwo & \weekthree~~ &  &  &  \\ \hline
Architectural Design   		&  &  &  & ~~\weekthree & \weekfour & \weektwo~~~ &  \\ \hline
Implementation		   		&  &  &  &  & ~~~\weektwo & \weekfour & \weekfour  \\ \hline
Testing 			   		&  &  &  &  &  &  & \weekfour \\ \hline
Documentation 		   		& ~~~\weektwo & \weekfour & \weekfour & \weekfour & \weekfour & \weekfour & \weekfour \\ \hline
\end{tabular}
\label{tab:timetableactivities1}
\end{table}

\begin{table}[ht]   %t means place on top, replace with b if you want to place at the bottom
\centering
\caption{Timetable of Activities for the months of August to December 2010} \vspace{0.25em}
\begin{tabular}{|p{2in}|c|c|c|c|c|} \hline
\centering Activities (2010) & Aug & Sep & Oct & Nov & Dec \\ \hline
Requirements Analysis	   	 &  &  &  &  & \\ \hline
Data Gathering			     &  &  &  &  & \\ \hline
Architectural Design   		 &  &  &  &  & \\ \hline
Implementation		   		 & \weekfour & \weekfour & \weekfour & \weekfour & \\ \hline
Testing 			   		 & \weekfour & \weekfour & \weekfour & \weekfour & \\ \hline
Documentation 		   		 & \weekfour & \weekfour & \weekfour & \weekfour & \weekfour \\ \hline
\end{tabular}
\label{tab:timetableactivities2}
\end{table}

\begin{table}[ht]   %t means place on top, replace with b if you want to place at the bottom
\centering
\caption{Timetable of Activities for the months of January to April 2013} \vspace{0.25em}
\begin{tabular}{|p{2in}|c|c|c|c|} \hline
\centering Activities (2010) & Jan & Feb & Mar & Apr \\ \hline
Requirements Analysis	   	 &  &  &  & \\ \hline
Data Gathering		         &  &  &  & \\ \hline
Architectural Design   		 &  &  &  & \\ \hline
Implementation		   		 & \weekfour & \weekfour &  & \\ \hline
Testing 			   		 & \weekfour & \weekfour & \weekfour & \weekone~~~~ \\ \hline
Documentation 		   		 & \weekfour & \weekfour & \weekfour & \weekone~~~~ \\ \hline
\end{tabular}
\label{tab:timetableactivities3}
\end{table}






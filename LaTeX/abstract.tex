%%%%%%%%%%%%%%%%%%%%%%%%%%%%%%%%%%%%%%%%%%%%%%%%%%%%%%%%%%%%%%%%%%%%%%%%%%%%%%%%%%%%%%%%%%%%%%%%%%%%%%
%
%   Filename    : abstract.tex 
%
%   Description : This file will contain your Research Description.
%                 
%%%%%%%%%%%%%%%%%%%%%%%%%%%%%%%%%%%%%%%%%%%%%%%%%%%%%%%%%%%%%%%%%%%%%%%%%%%%%%%%%%%%%%%%%%%%%%%%%%%%%%

\begin{abstract}
People use storytelling as a natural and familiar means of conveying information and experience to each other. During this interchange, people understand each other because we rely on a large body of shared common sense knowledge. But computers do not share this knowledge, causing a barrier in human-computer interaction and in applications requiring computers to generate coherent text. To support this task, computers must be provided with a usable knowledge about the basic relationships between concepts that we find everyday in our world. 

Picture Books is a story generation system that generates stories for children age 4 to 6. To achieve this, it uses a semantic ontology containing conceptual knowledge about objects, activities and their relationships in a child's daily life. But the task of building this knowledge base is tedious and time consuming, thus limiting the variants of stories and themes that Picture Books is able to generate. This research involves the development of a software tool that will automatically extract concepts and their relations from existing children's stories, and store these in a knowledge base that Picture Books and other NLP applications can utilize to do their tasks.  

\begin{flushleft}
\begin{tabular}{lp{4.25in}}
\hspace{-0.5em}\textbf{Keywords:}\hspace{0.25em} & language parsing and understanding, text analysis, semantic networks, natural language processing  \\
\end{tabular}
\end{flushleft}
\end{abstract}

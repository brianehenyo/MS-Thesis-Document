%%%%%%%%%%%%%%%%%%%%%%%%%%%%%%%%%%%%%%%%%%%%%%%%%%%%%%%%%%%%%%%%%%%%%%%%%%%%%%%%%%%%%%%%%%%%%%%%%%%%%%
%
%   Filename    : chapter_6.tex 
%
%   Description : This file will contain your Conclusion and Recommendations.
%                 
%%%%%%%%%%%%%%%%%%%%%%%%%%%%%%%%%%%%%%%%%%%%%%%%%%%%%%%%%%%%%%%%%%%%%%%%%%%%%%%%%%%%%%%%%%%%%%%%%%%%%%

\Section{Conclusion and Recommendations}
\label{sec:conclusionandreco}

This chapter discusses the conclusion of this research and provides recommendations and suggestions for future researches.

\subsection{Conclusion}
\label{sec:conclusion}

Based on the results obtained through the evaluation of the extractor, it was proven possible to extract new semantic relations from children's stories and feed them into Picture Books' ontology. However, the extractor was found to be inaccurate in doing so. Overall, it only got 0.36 as its precision, recall and F-measure scores for 3 MODIFIED stories. It even got lower scores for the RAW versions. It got 0.34, 0.32 and 0.33 for its precision, recall and F-measure, respectively. Therefore, the automatically extracted relations were mostly incorrect and the extractor was not able to extract all expected relations in a given text.

New relations were used accordingly and interchangeably, and new sentences were inserted. However, due to the limited themes currently present in Picture Books, not all extracted relations was used. There were also cases wherein all extracted relations won't be valid for Picture Books' use because the story where they came from are completely different from the existing themes. Another issue was Picture Books' way of accessing its ontology. Instead of using the relation names as reference, the relation category is used. Since each category has more than 1 relation associated, incorrect searches may still arise. Thus rendering the attempt to improve Picture Books' conceptual knowledge less remarkable. Lastly, it is not enough to just add new concepts and relations in the current ontology. Additional steps must be taken depending on how the ontology path you are trying to branch is accessed.

As for the extracted relations, their quality was greatly affected by the following:

\subsubsection{Stories}
\label{sec:stories}

Each story in the corpora has different characteristics. Some are lengthy while some are short. Some use a lot of complex sentence structures while others kept it simple. It all depends on the age group of the audience they are trying to reach. After evaluation, it is conclusive that as the sentence structures become more complex and the length of the story increases, the extractions get less accurate. It exposes a limitation on the templates used as they can only successfully handle simpler sentences and simpler manifestations of a relation in a text.

\subsubsection{Part-of-speech Tags}
\label{sec:pos}

The quality and accuracy of the part-of-speech tags supplied by GATE greatly affected the relations extracted. Because most of the extraction rules/templates mainly use part-of-speech tags in their annotation patterns, a slight mistake may cause the relation to have incorrect concepts or to have it not extracted at all. Here is a sample sentence to illustrate this scenario:

\begin{verse}
\itshape
Tigger takes a bath because he wants to be clean.
\end{verse}

One relation that can be extracted from this would be \textit{EventForGoalState(takes a bath,be clean)}. However, after numerous attempts in the application, that relation was extracted because \textit{clean} is tagged as a verb. The researcher's rule for \textit{EventForGoalState} requires the child concept to be an adjective for it to be called a desired state.

\subsubsection{Extraction rules}
\label{sec:templates}

Because the current set of extraction rules are generalized based on ConceptNet sentence patterns and the sentences present in the current corpora, there is a perceived limit in the capabilities of the extractor. And in an attempt to cover all sentence patterns with the least number of rules, exceptional cases may not be covered. Also, these rules cannot handle implied and inferred relations. If there are any implied or inferred ones in the text, the rules will not recognize them unless they were explicitly indicated after modification. Additionally, these rules do not have enough semantic information for most words. When a different sense of a word is used in a sentence, there is no way for the extractor to recognise it. It relies solely on the part-of-speech and named-entity tags to extract a relation. This deficiency in semantic information also causes an incorrect relation to be tagged to a concept pair. For example, \textit{PartOf} relations can be incorrectly tagged as an \textit{Owns} relation because of the similarity in templates used. Lastly, the templates were limited to extract relations from one or two adjacent sentences only. Most of the relations encountered in the gold standard span multiple sentences (more then two sentences), and this was not extensively considered in this approach.

\subsubsection{Indicators}
\label{sec:indicators}

The prevalent use of indicators in most of the extraction templates posed a limitation on the number and quality of extractions done. First, in most cases, indicators are not always used because of their formality. This also assumes that the concepts constituting a relation is within a sentence. If not, it is assumed that the second concept is in the next sentence, the subject pronoun referring to the first concept, and the whole thing signalled by an indicator.

Secondly, most relations identified in the gold standard were inferred or implied. Taking a look into the \textit{HasRole} relation again, the only instance did not have both concepts present in the story. In the relation \textit{HasRole(Miss Hen,teacher)}, the word \textit{teacher} was not found in the text. It was inferred as a role through the actions done by the character. Lastly, existing indicators are not enough. There's still a number of indicators not included in this study. This caused the only expected \textit{OftenNear} relation in \textit{Everybody Cries} to not be extracted. It was signalled by the word \textit{along} which is not part of the \textit{OftenNear} indicators.

\subsection{Recommendations}
\label{sec:reco}

Overall, the relation extractor was able to produce good enough relations to be used by any story generation system. The following are recommendations for future improvement:

\begin{itemize}
\item Improve the extraction rules. Incorporate as many patterns as possible. If allowable, run a big corpora through a machine learning tool that will learn all possible sentence patterns for each relation type.
\item Allow inferencing between relations as they are extracted. This will improve the compactness of the ontology.
\item Focus more on extracting event relations since they are not usually explicitly indicated in a span of text. This also constitutes the bulk of a story. Building an accurate cause-effect chain of events would be very beneficial for most creative text generation systems.
\item Look for a language resource that can supply accurate semantic information.
\item Since a gold standard was already utilized in evaluating this research, it would be beneficial to look for experts that can create different sets of gold standards for the different relations to improve the evaluation of the results. These sets could then be combined to form a comprehensive gold standard. Aside from this, have experts that can consistently evaluate the generated story after adding new relations. 
\end{itemize}

And because part of the evaluation involved the use of Picture Books in generating stories, some issues and limitations were encountered. The following are recommendations for Picture Books' future improvement:

\begin{itemize}
\item Since it is now possible to branch out in ontology searches, it would be advisable to keep track of previously chosen concepts that is expected to be used throughout the story. This will avoid conflicting details and make the output stories more coherent.
\item To be able to use all extracted relations, devise a way to automatically add new themes in Picture Books based on the themes found in the corpora. 
\item Instead of searching the ontology by category, use the relation names for more specific and accurate ontology access results.
\item Integrate VerbNet to make the character goals dynamic in creating sentences. This will provide information on what arguments are needed in a sentence for a certain verb.
\end{itemize}









